\documentclass[letterpaper,12pt]{article}
\usepackage[utf8]{inputenc}
\usepackage[T1]{fontenc}
\usepackage{tabularx} 
\usepackage{amsmath}  
\usepackage{graphicx} 
\usepackage[margin=1in,letterpaper]{geometry}
\usepackage[final]{hyperref} 
\usepackage{listings}
\usepackage[english]{babel}
\usepackage{xcolor}
\usepackage{minted}
\usepackage{mdframed}
\usemintedstyle{monokai}
\definecolor{bg}{rgb}{0.14,0.14,0.14}

\definecolor{mygray}{rgb}{0.4,0.4,0.4}
\definecolor{mygreen}{rgb}{0.40,0.70,0.30}
\definecolor{myorange}{rgb}{1.0,0.4,0}

\newcommand{\question}[1]{\paragraph{#1\\ \vspace{0.5cm}}}

\lstdefinestyle{C}
{
basicstyle=\footnotesize\sffamily\color{black},
commentstyle=\color{mygray},
breaklines=true,
frame=single,
numbers=left,
numbersep=5pt,
numberstyle=\tiny\color{mygray},
keywordstyle=\color{mygreen},
showspaces=false,
showstringspaces=false,
stringstyle=\color{myorange},
tabsize=2
}


\lstdefinestyle{BASH}
{
    backgroundcolor=\color{black},
    basicstyle=\scriptsize\color{white}\ttfamily
}



\begin{document}

\title{Systèmes Parallèles -- Lab. Session 2\\
\vspace{0.5cm}
Introduction au parallélisme en mémoire partagée
OpenMP}
\author{\textsc{Guillaume Leignier, Vincent Pasquereau, Thibaut Vercueil}}
\date{5 novembre 2015}
\maketitle
\tableofcontents

\newpage
\section{Évaluation des performances paralèlles}
\textit{L'ensemble du TP a été réalisé sur un processeur Intel i5 -- 4 coeurs.}

\paragraph{1.  Mesurer le temps d’exécution du code de chaque annexe modifiée en faisant varier n, le nombre de threads, de 1 à 8.}

Pour le code C, on a :\\

\begin{center}
\begin{tabular}{|c||c|c|c|c|c|c|c|c|}
  \hline
  Threads & 1 & 2 & 3 & 4 & 5 & 6 & 7 & 8 \\
  \hline
  Temps & 1.04 & 0.54 & 0.37 & 0.30 & 0.36 & 0.30 & 0.30 & 0.29 \\
  \hline 
\end{tabular}
\end{center}


\vspace{1cm}
Pour le code D, on a :\\ 

\begin{lstlisting}[style=BASH]
$ time -p cat AnnexeC.txt | ./b.out 
real 29.25
user 29.03
sys 0.20
\end{lstlisting}


\question{2. Pour l’annexe C, comparez vos résultats avec ceux obtenus au TP précédent avec les Pthreads.}


Voici une comparaison des résultats pour l'annexe C (le calcul de la somme) avec l'annexe A du TP1:


\begin{center}
\begin{tabular}{|c||c|c|c|c|c|c|c|c|}
  \hline
   Nombre de threads & 1 & 2 & 3 & 4 & 5 & 6 & 7 & 8 \\
  \hline
   Temps d'exécution annexe A (PThread) & 1.23 & 0.68 & 0.87 & 0.76 & 0.81 & 0.75 & 0.75 & 0.73 \\
  \hline 
  Temps d'exécution annexe C (OpenMP) & 1.04 & 0.54 & 0.37 & 0.30 & 0.36 & 0.30 & 0.30 & 0.29 \\
  \hline
 Différence (Pthread - OpenMP) & 0.19 & 0.14 & 0.5 & 0.46 & 0.45 & 0.45 & 0.45 & 0.44 \\
 \hline 
\end{tabular}
\end{center}

On remarque que l'utilisation de OpenMP apporte de meilleures performances pour la parallelisation.

En effet, une différence notable: environs 0.5 ms s'observe à partir de 3 threads. Cette différence correspond à une amélioration d'environs 135\%.

%TODO : Faire un joli graphique 

\question{Tracer la courbe du ratio $S(n) = \frac{real(1)}{real(n)}$ en fonction de n.}





\end{document}
